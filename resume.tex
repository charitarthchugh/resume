%%%%%%%%%%%%%%%%%%%%%%%%%%%%%%%%%%%%%%%
% Deedy - One Page Two Column Resume
% LaTeX Template
% Version 1.2 (16/9/2014)
%
% Original author:
% Debarghya Das (http://debarghyadas.com)
%
% Original repository:
% https://github.com/deedydas/Deedy-Resume
%
% IMPORTANT: THIS TEMPLATE NEEDS TO BE COMPILED WITH XeLaTeX
%
% This template uses several fonts not included with Windows/Linux by
% default. If you get compilation errors saying a font is missing, find the line
% on which the font is used and either change it to a font included with your
% operating system or comment the line out to use the default font.
% 
%%%%%%%%%%%%%%%%%%%%%%%%%%%%%%%%%%%%%%
% 
% TODO:
% 1. Integrate biber/bibtex for article citation under publications.
% 2. Figure out a smoother way for the document to flow onto the next page.
% 3. Add styling information for a "Projects/Hacks" section.
% 4. Add location/address information
% 5. Merge OpenFont and MacFonts as a single sty with options.
% 
%%%%%%%%%%%%%%%%%%%%%%%%%%%%%%%%%%%%%%
%
% CHANGELOG:
% v1.1:
% 1. Fixed several compilation bugs with \renewcommand
% 2. Got Open-source fonts (Windows/Linux support)
% 3. Added Last Updated
% 4. Move Title styling into .sty
% 5. Commented .sty file.
%
%%%%%%%%%%%%%%%%%%%%%%%%%%%%%%%%%%%%%%%
%
% Known Issues:
% 1. Overflows onto second page if any column's contents are more than the
% vertical limit
% 2. Hacky space on the first bullet point on the second column.
%
%%%%%%%%%%%%%%%%%%%%%%%%%%%%%%%%%%%%%%


\documentclass[]{resume-template}
\usepackage{fancyhdr}

\pagestyle{fancy}
\fancyhf{}

\begin{document}

%%%%%%%%%%%%%%%%%%%%%%%%%%%%%%%%%%%%%%
%
%     LAST UPDATED DATE
%
%%%%%%%%%%%%%%%%%%%%%%%%%%%%%%%%%%%%%%
    \lastupdated{}

%%%%%%%%%%%%%%%%%%%%%%%%%%%%%%%%%%%%%%
%
%     TITLE NAME
%
%%%%%%%%%%%%%%%%%%%%%%%%%%%%%%%%%%%%%%
    \namesection{Charitarth Chugh}{}{ \urlstyle{same}\href{http://charitarth.dev}{charitarth.dev} |
    \href{mailto:charitarth.chugh@gmail.com}{charitarth.chugh@gmail.com} | 475.434.6427}

%%%%%%%%%%%%%%%%%%%%%%%%%%%%%%%%%%%%%%
%
%     COLUMN ONE
%
%%%%%%%%%%%%%%%%%%%%%%%%%%%%%%%%%%%%%%

    \begin{minipage}[t]{0.33\textwidth}

%%%%%%%%%%%%%%%%%%%%%%%%%%%%%%%%%%%%%%
%     EDUCATION
%%%%%%%%%%%%%%%%%%%%%%%%%%%%%%%%%%%%%%

        \section{Education}\label{sec:education}

        \subsection{UConn\\ Computer Science}\label{subsec:uconn-school-of-engineering}

        \descript{GPA N\/A}
        \location{Expected Graduation: Spring 2025}
        \sectionsep{}


%%%%%%%%%%%%%%%%%%%%%%%%%%%%%%%%%%%%%%
%     Volunteering and Activities
%%%%%%%%%%%%%%%%%%%%%%%%%%%%%%%%%%%%%%


        \section{Activities}\label{sec:activities}

        \subsection{UConn AI Club}\label{subsec:uconn-ai-club}
        \descript{Secretary: 2021 - 2022}
        AI Club does workshops,\\showcases, and builds projects\\ around deep learning.\\
        Led and organized meetings.
        Helped revive \\the club for Fall 2021 semester
        \vspace{\topsep}

        \subsection{Hack Club}\label{subsec:hack-club}
        \descript{President: 2020 - 2021}
        \descript{Vice President: 2019 - 2020}
        \descript{Board: 2018 - 2019}
        Explore, share, and learn\\ about
        new technologies, with \\
        an emphasis on programming\\
        and computer hardware.\\ Led and organized weekly meetings.
        \vspace{\topsep}

        \subsection{Volunteering}\label{subsec:volunteering}
        \descript{Time: 220+ hours}
        Volunteered at Trumbull \\
        Public Library for their\\
        Summer Reading Program
        \sectionsep{}
%%%%%%%%%%%%%%%%%%%%%%%%%%%%%%%%%%%%%%
%     SKILLS
%%%%%%%%%%%%%%%%%%%%%%%%%%%%%%%%%%%%%%


        \section{Skills}\label{sec:skills}

        \subsection{Programming}\label{subsec:programming}
        \location{Comfortable:}
        Java \textbullet{} Flutter \textbullet{} Git \textbullet{} GitHub\\
        Python \textbullet{} Pandas \textbullet{} Plotly\\
        Matplotlib \textbullet{} PyTorch \textbullet Numpy\\
        \location{Familiar:}
        Linux \textbullet{} Shell \textbullet{}  \LaTeX{} \textbullet{} Kotlin\\
        Docker \textbullet{}  HTML \textbullet{}  CSS
        \sectionsep{}

        \subsection{Languages}
%\location{Languages :}
        English \textbullet{} Hindi (Speaker)\\
        `Spanish (Basic)
%\location{}
%%%%%%%%%%%%%%%%%%%%%%%%%%%%%%%%%%%%%%
%     LINKS
%%%%%%%%%%%%%%%%%%%%%%%%%%%%%%%%%%%%%%


        \section{Links}\label{sec:links}
        GitHub:// \href{https://github.com/charitarthchugh}{\textbf {charitarthchugh}} \\
        LinkedIn:// \href{https:///www.linkedin.com/in/charitarth}{\textbf {charitarth}} \\
        Twitter:// \href{https://twitter.com/charitarthchugh}{\textbf{@charitarthchugh}}\\
        Kaggle:// \href{https://kaggle.com/charitarth}{\textbf{charitarth}}\\
        Medium:// \href{https://medium.com/@charitarth.chugh}{\textbf{@charitarth.chugh}}\\
%%%%%%%%%%%%%%%%%%%%%%%%%%%%%%%%%%%%%%
%
%     COLUMN TWO
%
%%%%%%%%%%%%%%%%%%%%%%%%%%%%%%%%%%%%%%

    \end{minipage}
    \hfill
    \begin{minipage}[t]{0.66\textwidth}

%%%%%%%%%%%%%%%%%%%%%%%%%%%%%%%%%%%%%%
%     Projects
%%%%%%%%%%%%%%%%%%%%%%%%%%%%%%%%%%%%%%

        \section{Projects}\label{sec:projects}

        \subsection{OpinionMining}\label{subsec:opinionmining}
        \descript{Natural Language Processing}
        \location{October 2021 - Present}
        \vspace{\topsep}
        \begin{tightemize}
            \item Opinion Mining, also known as Aspect-based Sentiment Analysis (ABSA) is a subfield of sentiment analysis
            where a model detects one or more entities, aspects and opinions within a textual input.
            \item Created a BERT model with a custom head that better detects implicit opinion within a given input

        \end{tightemize}

        \subsection{Exo-EDA}\label{subsec:exo-eda}
        \descript{Data Analysis}
        \location{July - August 2021}
        \vspace{\topsep} % Hacky fix for awkward extra vertical space
        \begin{tightemize}
            \item In-depth analysis of exoplanet data from the NASA Exoplanet Archive, using Pandas, NumPy,
            Seaborn, and Matplotlib.
            \item Retrieved data using the TAP API to allow users to always have the latest data when running the Jupyter Notebook
            \item Cleaned large amounts of data for a 36\% reduction in memory usage
            \item Identified planets that reside in the habitable zone of their host star and found that our solar system is a
            relative anomaly, as the majority of solar systems only host one or two planets
        \end{tightemize}

        \subsection{Pulsar Identification}\label{subsec:pulsar-identification}
        \descript{Machine Learning}
        \location{June 2020}
        %\vspace{\topsep} % Hacky fix for awkward extra vertical space
        \begin{tightemize}
            \item Created a 97\% accurate classifier using a custom Logistic Regression model made with Numpy, Pandas, and PyTorch for the classification of pulsars in the HTRU1 dataset
            %\item Used Numpy, Pandas, and PyTorch to create a custom Logistic Regression model.
        \end{tightemize}
%\sectionsep
%\sectionsep


%\sectionsep
        \sectionsep{}

%%%%%%%%%%%%%%%%%%%%%%%%%%%%%%%%%%%%%%
%     RESEARCH
%%%%%%%%%%%%%%%%%%%%%%%%%%%%%%%%%%%%%%

%\section{Research}\label{sec:research}
%\runsubsection{Cornell Robot Learning Lab}
%\descript{| Researcher}
%\location{Jan 2014 – Jan 2015 | Ithaca, NY}
%Worked with \textbf{\href{http://www.cs.cornell.edu/~ashesh/}{Ashesh Jain}} and \textbf{\href{http://www.cs.cornell.edu/~asaxena/}{Prof Ashutosh Saxena}} to create \textbf{PlanIt}, a tool which  learns from large scale user preference feedback to plan robot trajectories in human environments.  
%\sectionsep

%\runsubsection{Cornell Phonetics Lab}
%\descript{| Head Undergraduate Researcher}
%\location{Mar 2012 – May 2013 | Ithaca, NY}
%Led the development of \textbf{QuickTongue}, the first ever breakthrough tongue-controlled game with \textbf{\href{http://conf.ling.cornell.edu/~tilsen/}{Prof Sam Tilsen}} to aid in Linguistics research. 
%\sectionsep

%%%%%%%%%%%%%%%%%%%%%%%%%%%%%%%%%%%%%%
%     Cetifications and Awards
%%%%%%%%%%%%%%%%%%%%%%%%%%%%%%%%%%%%%%


        \section{Certifications and Awards}\label{sec:certifications-and-awards}
        \runsubsection{\href{https://stamforddatascience.com/hackathon}{Coindesk x TradeBlock Crypto Hackathon}}
        \descript{| 1st Place }
        \location{ December 2019 - January 2020 }
        \begin{tightemize}
            \item With a  5-person team developed a custom momentum based algorithm that detected rises and falls within
            Bitcoin and Ethereum prices with a custom load factor to detect volumes of trades
            \item We faced problems with the data such as invalid/null values and high volatility which needed to be
            accounted for
        \end{tightemize}
        \sectionsep{}
        \runsubsection{Zero To GANs}
        \descript{|  Certification}
        \location{ July 2020}
        \begin{tightemize}
            \item Given for successful completion of "Deep Learning with Pytorch: Zero to GANs",
            a six-week online course offered in collaboration by FreeCodeCamp and JovianAI.
            \item Represents about 60 hours of coursework, which required doing weekly assignments, watching
            lectures, a course project and a Kaggle Competition
        \end{tightemize}
        \sectionsep{}

        \runsubsection{Zero To Pandas}
        \descript{|  Certification}
        \location{August 2021}
        \begin{tightemize}
            \item Given for successful completion of "Data Analysis with Python: Zero to Pandas",
            a six-week online course offered in collaboration by FreeCodeCamp and JovianAI.
            \item Represents about 60 hours of coursework, which required doing weekly assignments, watching lectures, and a course project.
        \end{tightemize}
        \sectionsep{}

%%%%%%%%%%%%%%%%%%%%%%%%%%%%%%%%%%%%%%
%     PUBLICATIONS
%%%%%%%%%%%%%%%%%%%%%%%%%%%%%%%%%%%%%%




    \end{minipage}
\end{document}
