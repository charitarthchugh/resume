%%%%%%%%%%%%%%%%%%%%%%%%%%%%%%%%%%%%%%%
% Deedy - One Page Two Column Resume
% LaTeX Template
% Version 1.2 (16/9/2014)
%
% Original author:
% Debarghya Das (http://debarghyadas.com)
%
% Original repository:
% https://github.com/deedydas/Deedy-Resume
%
% IMPORTANT: THIS TEMPLATE NEEDS TO BE COMPILED WITH XeLaTeX
%
% This template uses several fonts not included with Windows/Linux by
% default. If you get compilation errors saying a font is missing, find the line
% on which the font is used and either change it to a font included with your
% operating system or comment the line out to use the default font.
% 
%%%%%%%%%%%%%%%%%%%%%%%%%%%%%%%%%%%%%%
% 
% TODO:
% 1. Integrate biber/bibtex for article citation under publications.
% 2. Figure out a smoother way for the document to flow onto the next page.
% 3. Add styling information for a "Projects/Hacks" section.
% 4. Add location/address information
% 5. Merge OpenFont and MacFonts as a single sty with options.
% 
%%%%%%%%%%%%%%%%%%%%%%%%%%%%%%%%%%%%%%
%
% CHANGELOG:
% v1.1:
% 1. Fixed several compilation bugs with \renewcommand
% 2. Got Open-source fonts (Windows/Linux support)
% 3. Added Last Updated
% 4. Move Title styling into .sty
% 5. Commented .sty file.
%
%%%%%%%%%%%%%%%%%%%%%%%%%%%%%%%%%%%%%%%
%
% Known Issues:
% 1. Overflows onto second page if any column's contents are more than the
% vertical limit
% 2. Hacky space on the first bullet point on the second column.
%
%%%%%%%%%%%%%%%%%%%%%%%%%%%%%%%%%%%%%%


\documentclass[]{resume-template}
\usepackage{fancyhdr}
 
\pagestyle{fancy}
\fancyhf{}
 
\begin{document}

%%%%%%%%%%%%%%%%%%%%%%%%%%%%%%%%%%%%%%
%
%     LAST UPDATED DATE
%
%%%%%%%%%%%%%%%%%%%%%%%%%%%%%%%%%%%%%%
\lastupdated{}

%%%%%%%%%%%%%%%%%%%%%%%%%%%%%%%%%%%%%%
%
%     TITLE NAME
%
%%%%%%%%%%%%%%%%%%%%%%%%%%%%%%%%%%%%%%
\namesection{Charitarth Chugh}{}{ \urlstyle{same}\href{http://charitarth.codes}{charitarth.codes} |
\href{mailto:charitarth.chugh@gmail.com}{charitarth.chugh@gmail.com} | 475.434.6427 
}

%%%%%%%%%%%%%%%%%%%%%%%%%%%%%%%%%%%%%%
%
%     COLUMN ONE
%
%%%%%%%%%%%%%%%%%%%%%%%%%%%%%%%%%%%%%%

\begin{minipage}[t]{0.33\textwidth} 

%%%%%%%%%%%%%%%%%%%%%%%%%%%%%%%%%%%%%%
%     EDUCATION
%%%%%%%%%%%%%%%%%%%%%%%%%%%%%%%%%%%%%%

\section{Education}


\subsection{Trumbull High}
\descript{GPA (Weighted): 3.73}
\location{June 2021 | Trumbull, CT}
\sectionsep{}



%%%%%%%%%%%%%%%%%%%%%%%%%%%%%%%%%%%%%%
%     Volunteering and Activities
%%%%%%%%%%%%%%%%%%%%%%%%%%%%%%%%%%%%%%

\section{Activities}\label{sec:activities}

\subsection{Hack Club}
\descript{President: 2020 - 2021}
\descript{Vice President: 2019 - 2020}
\descript{Board: 2018 - 2019} 
Explore, share, and learn\\ about
new technologies, with \\
an emphasis on programming\\
and computer hardware.\\ Led and organized weekly meetings.
\vspace{\topsep}
\subsection{COLT}
\descript{Secretary: 2019 - 2021}
COLT is a foreign language\\ poetry competition in CT.\\ Helped in Hindi poetry and organizing meetings.
\sectionsep{}

\subsection{Volunteering}\label{subsec:volunteering}
\descript{Time: 220+ hours}
Volunteered at Trumbull \\
Public Library for their\\
Summer Reading Program
\sectionsep{}


\subsection{Tennis}
\descript{Junior Varsity: 2019 - 2020}
%%%%%%%%%%%%%%%%%%%%%%%%%%%%%%%%%%%%%%
%     SKILLS
%%%%%%%%%%%%%%%%%%%%%%%%%%%%%%%%%%%%%%

\section{Skills}\label{sec:skills}

\subsection{Programming}\label{subsec:programming}
\location{Comfortable:}
Java \textbullet{} Flutter \textbullet{} Git \textbullet{} GitHub\\
NumPy \textbullet{} Pandas \textbullet{} Plotly\\
Matplotlib \textbullet{} PyTorch \\
\location{Familiar:}
Linux \textbullet{} Shell \textbullet{}  \LaTeX{} \textbullet{} Kotlin\\
Docker \textbullet{}  HTML \textbullet{}  CSS
\sectionsep{}

\subsection{Languages}
%\location{Languages :} 
English \textbullet{} Hindi (Speaker)\\ 
`Spanish (Basic) 
%\location{}
%%%%%%%%%%%%%%%%%%%%%%%%%%%%%%%%%%%%%%
%     LINKS
%%%%%%%%%%%%%%%%%%%%%%%%%%%%%%%%%%%%%%

\section{Links}\label{sec:links}
GitHub:// \href{https://github.com/charitarthchugh}{\textbf {charitarthchugh}} \\
Linkedin:// \href{https:///www.linkedin.com/in/charitarth}{\textbf {charitrth}} \\
JovianAI:// \href{https://jovian.ai/charitarthchugh}{\textbf{charitarthchugh}}\\
Kaggle:// \href{https://kaggle.com/charitarth}{\textbf{charitarth}}\\
Medium:// \href{https://medium.com/@charitarth.chugh}{\textbf{@charitarth.chugh}}
%%%%%%%%%%%%%%%%%%%%%%%%%%%%%%%%%%%%%%
%
%     COLUMN TWO
%
%%%%%%%%%%%%%%%%%%%%%%%%%%%%%%%%%%%%%%

\end{minipage} 
\hfill
\begin{minipage}[t]{0.66\textwidth} 

%%%%%%%%%%%%%%%%%%%%%%%%%%%%%%%%%%%%%%
%     Projects
%%%%%%%%%%%%%%%%%%%%%%%%%%%%%%%%%%%%%%

\section{Projects}\label{sec:projects}
\runsubsection{Solar India}
\descript{| Data Analysis}
\location{December 2020}
\vspace{\topsep} % Hacky fix for awkward extra vertical space
\begin{tightemize}
\item In-depth analysis of solar energy generation and sensor data from a solar plant in India, using Pandas, Numpy, Seaborn, and Matplotlib.
\item Identified inefficeincies at the plant, such as poorly functioning hardware.  
\end{tightemize}

\runsubsection{Pulsar Idenfication}
\descript{| Machine Learning}
\location{June 2020}
\vspace{\topsep} % Hacky fix for awkward extra vertical space
\begin{tightemize}
\item Created a 97\% accurate classifier using a custom Logistic Regression model made with Numpy, Pandas, and PyTorch for the classification of pulsars in the HTRU1 dataset
%\item Used Numpy, Pandas, and PyTorch to create a custom Logistic Regression model. 
\end{tightemize}
%\sectionsep

\runsubsection{Portfolio Showcase}
\descript{| Frontend}
\location{2020}
%\vspace{\topsep} % Hacky fix for awkward extra vertical space
\begin{tightemize}
\item Designed and developed a responsive website using Flutter SDK
\item Implemented Firebase Flutter integration and used the VelocityX framework for Flutter
\end{tightemize}
%\sectionsep


%\sectionsep
\sectionsep{}

%%%%%%%%%%%%%%%%%%%%%%%%%%%%%%%%%%%%%%
%     RESEARCH
%%%%%%%%%%%%%%%%%%%%%%%%%%%%%%%%%%%%%%

%\section{Research}\label{sec:research}
%\runsubsection{Cornell Robot Learning Lab}
%\descript{| Researcher}
%\location{Jan 2014 – Jan 2015 | Ithaca, NY}
%Worked with \textbf{\href{http://www.cs.cornell.edu/~ashesh/}{Ashesh Jain}} and \textbf{\href{http://www.cs.cornell.edu/~asaxena/}{Prof Ashutosh Saxena}} to create \textbf{PlanIt}, a tool which  learns from large scale user preference feedback to plan robot trajectories in human environments.  
%\sectionsep

%\runsubsection{Cornell Phonetics Lab}
%\descript{| Head Undergraduate Researcher}
%\location{Mar 2012 – May 2013 | Ithaca, NY}
%Led the development of \textbf{QuickTongue}, the first ever breakthrough tongue-controlled game with \textbf{\href{http://conf.ling.cornell.edu/~tilsen/}{Prof Sam Tilsen}} to aid in Linguistics research. 
%\sectionsep

%%%%%%%%%%%%%%%%%%%%%%%%%%%%%%%%%%%%%%
%     Cetifications and Awards
%%%%%%%%%%%%%%%%%%%%%%%%%%%%%%%%%%%%%%
\section{Certifications and Awards}

\runsubsection{Zero To GANs}
\descript{|  Certification}
\location{ July 2020}
\begin{tightemize}
\item Given for successful completion of "Deep Learning with Pytorch: Zero to GANs",
             a six-week online course offered in collaboration by FreeCodeCamp and JovianAI.
\item Represents about 60 hours of coursework, which required doing weekly assignments, watching lectures, and a course project
\item Gained fundamental math skills that are utilized in machine learning. 
\end{tightemize}
\sectionsep{}

\runsubsection{Zero To Pandas}
\descript{|  Certification}
\location{ October 2020}
\begin{tightemize}
\item Given for successful completion of "Data Analysis with Python: Zero to Pandas", 
            a six-week online course offered in collaboration by FreeCodeCamp and JovianAI.
\item Represents about 60 hours of coursework, which required doing weekly assignments, watching lectures, and a course project.
\end{tightemize}
\sectionsep{}

\runsubsection{THS Golden Eagle}
\descript{|  Award}
\location{ June 2019 \& June 2020 }
\begin{tightemize}
\item Given to Trumbull High students who have volunteered more than 100 hours in a year.
\item Awarded 2 times.
\end{tightemize}
\sectionsep{}

\runsubsection{\href{https://codein.withgoogle.com}{Google Code In}}
\descript{|  Participation}
\location{ December 2019 - January 2020 }
\begin{tightemize}
\item This was a coding competition hosted by Google that introduced pre-university students to open source software development
\end{tightemize}
\sectionsep{}

%%%%%%%%%%%%%%%%%%%%%%%%%%%%%%%%%%%%%%
%     PUBLICATIONS
%%%%%%%%%%%%%%%%%%%%%%%%%%%%%%%%%%%%%%




\end{minipage} 
\end{document}  \documentclass[]{article}
