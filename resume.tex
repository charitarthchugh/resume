%!TeX root = resume.tex
%%%%%%%%%%%%%%%%%%%%%%%%%%%%%%%%%%%%%%%
% Deedy - One Page Two Column Resume
% LaTeX Template
% Version 1.2 (16/9/2014)
%
% Original author:
% Debarghya Das (http://debarghyadas.com)
%
% Original repository:
% https://github.com/deedydas/Deedy-Resume
%
% IMPORTANT: THIS TEMPLATE NEEDS TO BE COMPILED WITH XeLaTeX
%
% This template uses several fonts not included with Windows/4


% default. If you get compilation errors saying a font is missing, find the line
% on which the font is used and either change it to a font included with your
% operating system or comment the line out to use the default font.
% 
%%%%%%%%%%%%%%%%%%%%%%%%%%%%%%%%%%%%%%
% 
% TODO:
% 1. Integrate biber/bibtex for article citation under publications.
% 2. Figure out a smoother way for the document to flow onto the next page.
% 3. Add styling information for a "Projects/Hacks" section.
% 4. Add location/address information
% 5. Merge OpenFont and MacFonts as a single sty with options.
% 
%%%%%%%%%%%%%%%%%%%%%%%%%%%%%%%%%%%%%%
%
% CHANGELOG:
% v1.1:
% 1. Fixed several compilation bugs with \renewcommand
% 2. Got Open-source fonts (Windows/Linux support)
% 3. Added Last Updated
% 4. Move Title styling into .sty
% 5. Commented .sty file.
%
%%%%%%%%%%%%%%%%%%%%%%%%%%%%%%%%%%%%%%%
%
% Known Issues:
% 1. Overflows onto second page if any column's contents are more than the
% vertical limit
% 2. Hacky space on the first bullet point on the second column.
%
%%%%%%%%%%%%%%%%%%%%%%%%%%%%%%%%%%%%%%


\documentclass[]{resume-template}
\usepackage{fancyhdr}

\pagestyle{fancy}
\fancyhf{}

\begin{document}

%%%%%%%%%%%%%%%%%%%%%%%%%%%%%%%%%%%%%%
%
%     LAST UPDATED DATE
%
%%%%%%%%%%%%%%%%%%%%%%%%%%%%%%%%%%%%%%
\lastupdated{}

%%%%%%%%%%%%%%%%%%%%%%%%%%%%%%%%%%%%%%
%
%     TITLE NAME
%
%%%%%%%%%%%%%%%%%%%%%%%%%%%%%%%%%%%%%%
\namesection{Charitarth Chugh}{}{\href{mailto:contact@charitarth.dev}{contact@charitarth.dev} |
	\urlstyle{same}\href{https://charitarth.dev}{charitarth.dev} | 475.434.6427}

%%%%%%%%%%%%%%%%%%%%%%%%%%%%%%%%%%%%%%
%
%     COLUMN ONE
%
%%%%%%%%%%%%%%%%%%%%%%%%%%%%%%%%%%%%%%

\begin{minipage}[t]{0.33\textwidth}

	%%%%%%%%%%%%%%%%%%%%%%%%%%%%%%%%%%%%%%
	%     EDUCATION
	%%%%%%%%%%%%%%%%%%%%%%%%%%%%%%%%%%%%%%

	\section{Education}\label{sec:education}

	\subsection{UConn\\
		Mathematics-Statistics}\label{subsec:uconn}
	\location{Minor: Computer Science}
	\location{Expected Graduation: May 2025}
	\vspace{\topsep}

	\location{Relevant Coursework}\label{subsec:coursework}
	Machine Learning \textbullet{} Data Science \\
	Data Structures \& Algorithms\\
	Probability \textbullet{} Systems Programming\\
	Applied Linear Regression
	\sectionsep{}

	%         \subsection{Teaching}\label{subsec:teaching}
	% %    \location{Upcoming (Spring 2023):}
	%         CSE 4095/5097: Introduction to Transformers Architecture
	%%%%%%%%%%%%%%%%%%%%%%%%%%%%%%%%%%%%%%
	%     SKILLS
	%%%%%%%%%%%%%%%%%%%%%%%%%%%%%%%%%%%%%%


	\section{Skills}\label{sec:skills}
	\subsection{Programming}\label{subsec:programming}
	\location{Python:}
	NumPy\textbullet{} Pandas\textbullet{} Polars\textbullet{} Matplotlib\\
	FastAPI \textbullet{} SQLAlchemy \textbullet{} Flask
	\newline{}\location{Frontend:}
	Flutter \textbullet{} React


	\vspace*{1pt}\subsection{Machine Learning}\label{subsec:mltools}
	% \location{Concepts:}
	% Attention\textbullet{}
	\location{Tools:}
	PyTorch \textbullet{} Transformers \textbullet{} scikit-learn \\
	XGBoost \textbullet{} Albumentations

	\vspace*{1pt}\subsection{Development}
	GitHub Actions\textbullet{}  Linux \textbullet{} Bash \textbullet{} Containers (Docker, Podman)

	% \subsection{Programming}\label{subsec:programming}
	% \location{Python:}
	% PyTorch \textbullet{} Transformers \textbullet{} \\
	% Plotly \textbullet{} Matplotlib \textbullet{} Pandas \textbullet{}\\
	% NumPy \textbullet{} FastAPI \textbullet SQLAlchemy \\ %\textbullet{}% Flask \textbullet{} Jinja2\\
	% \location{Other:}
	% Flutter \textbullet{} Git \textbullet{} GitHub \textbullet{} SQLite \textbullet{}\\
	% Linux \textbullet Docker \textbullet Podman \textbullet {} CI/CD \\
	% JavaScript \textbullet{} NodeJS \textbullet{} React \textbullet{} GCP\\
	% \location{Familiar:}
	% R \textbullet{} Java \textbullet{} Kotlin \textbullet{} Bash \textbullet{} Fish \textbullet{} HTML \textbullet{}\\
	% CSS \textbullet{} \LaTeX \textbullet{} OpenAPI \textbullet{} AWS %\textbullet{} GCP
	% \vspace{\topsep}

	% \subsection{Languages}
	% English \textbullet{} Hindi (Speaker)\\
	% Spanish (Basic)
	%\location{}
	%%%%%%%%%%%%%%%%%%%%%%%%%%%%%%%%%%%%%%
	%     Volunteering and Activities
	%%%%%%%%%%%%%%%%%%%%%%%%%%%%%%%%%%%%%%


	\section{Activities}\label{sec:activities}

	\subsection{UConn AI Club}\label{subsec:uconn-ai-club}
	\descript{President: 2023-2025}
	\textbullet{} AI Club does workshops,\\showcases, and projects \\around deep learning.\\
	\textbullet{} Co-ordinated and lead weekly meetings, with topics such as PyTorch, Apache Spark, CNNs\\
	% \textbullet{} Spearheaded the Special Projects Group to do open-source contributions to LiquidPrep, an organization helping farmers

	\vspace{\topsep}


	% \subsection{Hack Club}\label{subsec:hack-club}
	% \descript{President: 2020 - 2021}
	% \descript{Vice President: 2019 - 2020}
	% \descript{Board: 2018 - 2019}
	% Explore, share, and learn\\ about
	% new technologies, with \\
	% an emphasis on programming\\
	% and computer hardware.\\ Led and organized weekly meetings.
	% \vspace{\topsep}

	% \subsection{Volunteering}\label{subsec:volunteering}
	% \descript{Time: 220+ hours}
	% Volunteered at Trumbull \\
	% Public Library for their\\
	% Summer Reading Program
	% \sectionsep{}


	%%%%%%%%%%%%%%%%%%%%%%%%%%%%%%%%%%%%%%
	%     LINKS
	%%%%%%%%%%%%%%%%%%%%%%%%%%%%%%%%%%%%%%


	\section{Links}\label{sec:links}
	GitHub:// \href{https://github.com/charitarthchugh}{\textbf {charitarthchugh}} \\
	LinkedIn:// \href{https:///www.linkedin.com/in/charitarth}{\textbf {charitarth}} \\
	Twitter:// \href{https://twitter.com/charitarthchugh}{\textbf{@charitarthchugh}}\\
	Kaggle:// \href{https://kaggle.com/charitarth}{\textbf{charitarth}}\\
	Medium:// \href{https://medium.com/@charitarth.chugh}{\textbf{@charitarth.chugh}}\\
	%%%%%%%%%%%%%%%%%%%%%%%%%%%%%%%%%%%%%%
	%
	%     COLUMN TWO
	%
	%%%%%%%%%%%%%%%%%%%%%%%%%%%%%%%%%%%%%%

\end{minipage}
\hfill
\begin{minipage}[t]{0.66\textwidth}
	%%%%%%%%%%%%%%%%%%%%%%%%%%%%%%%%%%%%%%
	%     Work Experience
	%%%%%%%%%%%%%%%%%%%%%%%%%%%%%%%%%%%%%%

	\section{Work Experience}\label{sec:Work Experience}
	\runsubsection{Protection Shield}\label{subsec:protectionshield}
	\descript{| MLE, Freelance}
	\location{September 2023 - May 2024}
	\vspace{\topsep}
	\begin{tightemize}
		% \item Protection Shield is a startup working on intelligent network firewall systems
		\item Worked on their AI team to build federated learning models to detect network attacks.
		\item Built baseline models on publically available datasets such as NF-UQ-NIDS v2.
	\end{tightemize}
	%%%%%%%%%%%%%%%%%%%%%%%%%%%%%%%%%%%%%%
	%     Projects
	%%%%%%%%%%%%%%%%%%%%%%%%%%%%%%%%%%%%%%

	\section{Projects}\label{sec:projects}
	\runsubsection{Energy Justice Mapping Tool}\label{subsec:exo-eda}
	\descript{| Data Science}
	\location{July - August 2024}
	%\vspace{\topsep} % Hacky fix for awkward extra vertical space
	\begin{tightemize}
		% \item Building an interactive web dashboard to empower communities and stakeholders to address energy distribution inequities across Connecticut      % \begin{itemize}
		% %     % \item Found exoplanets that orbit multiple stars
		% %     % \item
		% % \end{itemize}
		% % \item Retrieved data using a domain-specific API
		% \item Working in a multidisciplinary team, combining expertise in civil engineering, environmental engineering, and data science
		% \item Successfully utilized a suite of over 5+ data sources to facilitate real-time data access and enhance analytical capabilities.
		% % \item Leading
		% % \item
		\item The Energy Justice Mapping Tool is a solution to find areas which lack equitable access to energy infrastructure.
		\item Worked in a multidisciplinary team whose proposal for a \$2,500 grant was selected for the Clean Energy \& Sustainability Innovation Program 2024
		\item Responsible for the integration and real-time analysis of geospatial data from 5+ data sources using GeoPandas
		\item Presented to White House officials and directors of Eversource Energy at the Clean Energy Summit 2024
	\end{tightemize}
	% \runsubsection{Neatbot}\label{subsec:neatbot}
	% \descript{| MLOps}
	% \location{June 2022-July 2022}
	% \begin{tightemize}
	%     \item Created a Discord Bot that detects code languages in a code block and replies with the correct syntax highlighting
	%     \item Deployed to \textbf{GCP} using Container Registry and Compute Engine
	%     \item Achieved a less than 10 second end to end response time
	%     % \item Working on creating a new model that can detect 4x the languages, while also having faster inference. 
	% \end{tightemize}
	\runsubsection{SparseInst}\label{subsec:neatbot}
	\descript{| Computer Vision}
	\location{September-December 2024}
	\begin{tightemize}
		\item Replicating the results of Sparse Instance Activation for Real-Time Instance Segmentation by Cheng et al. (2022), published at CVPR 2022
		\item Implementing SparseInst on ResNet-50 backbone
		\item Utilizing PyTorch Lightning, FiftyOne, WandB and Optuna for model training, testing, and evaluation.
		\item Assessing model performance by comparing segmentation accuracy and speed metrics, such as Average Precision and FPS, against the original results.
	\end{tightemize}
	\runsubsection{Bookie}\label{subsec: Bookie}
	\descript{| Full Stack}
	\location{May 2022-July 2022}
	% \vspace{\topsep}
	\begin{tightemize}
		\item Created a cross-platform bookmark manager using FastAPI, SQLite \& Flutter
		\item Led a 5-person cross-functional team integrating frontend, backend, and database
		\item Setup GitHub Actions for code cleanup
		\item Developed CLI interface, API, daemon and facilitated Python packaging
		\item Rewrote database for faster writes and updates, using recursive SQL database structure. \\
		% while also contributing to the creation and design of the Flutter application. In my role as the project manager, I supervised the creation of new features and encouraged standards that promote future
		% maintainability
		%        \item Supervised the development of new features and encouraged standards that promote future maintainability
	\end{tightemize}
	% \vspace{\topsep}

	%     \runsubsection{OpinionMining}\label{subsec:opinionmining}
	%     \descript{| Natural Language Processing}
	%     \location{October 2021 - Current}
	%     %\vspace{\topsep}
	%     \begin{tightemize}
	%         \item Opinion Mining, also known as Aspect-based Sentiment Analysis (ABSA) is a subfield of sentiment analysis
	%         where a model detects one or more entities, aspects and opinions within a textual input.
	%         \item Created a custom Bert-based model that better detects implicit opinion within a given input
	%     \end{tightemize}
	%     \vspace{\topsep}

	% \runsubsection{Spam Classification}\label{subsec:pulsar-identification}
	% \descript{| Machine Learning}
	% \location{June 2020}
	% %\vspace{\topsep} % Hacky fix for awkward extra vertical space
	% \begin{tightemize}
	%     \item Created a 97\% accurate classifier using a custom Logistic Regression model made with Numpy, Pandas, and PyTorch for the classification of pulsars in the HTRU1 dataset
	%     %\item Used Numpy, Pandas, and PyTorch to create a custom Logistic Regression model.
	% \end{tightemize}


	% \vspace{\topsep}

	% \runsubsection{Pulsar Identification}\label{subsec:pulsar-identification}
	% \descript{| Machine Learning}
	% \location{June 2020}
	% %\vspace{\topsep} % Hacky fix for awkward extra vertical space
	% \begin{tightemize}
	%     \item Created a 97\% accurate classifier using a custom Logistic Regression model made with Numpy, Pandas, and PyTorch for the classification of pulsars in the HTRU1 dataset
	%     %\item Used Numpy, Pandas, and PyTorch to create a custom Logistic Regression model.
	% \end{tightemize}
	% \sectionsep


	%\sectionsep


	%\sectionsep
	% \sectionsep{}

	%%%%%%%%%%%%%%%%%%%%%%%%%%%%%%%%%%%%%%
	%     RESEARCH
	%%%%%%%%%%%%%%%%%%%%%%%%%%%%%%%%%%%%%%


	\section{Research}\label{sec:research}
	\runsubsection{Undergraduate Researcher}
	% \descript{Undergraduate Researcher}
	\location{November 2022 - Present}
	\textbullet{} Developed and optimized machine learning models for embedded devices for object detection tasks under the guidance of \href{https://caiwending.cse.uconn.edu/}{Dr. Caiwen Ding} \\
	%    Working under  to develop optimized machine learning models for object detection tasks on embedded devices, using techniques such as sparsity and quantization, achieving 50mAP.
	\textbullet{} Working under Dr.Derek Aguiar to create tabular models that predict motion outcomes in legal cases using TabTransfomers.

	%%%%%%%%%%%%%%%%%%%%%%%%%%%%%%%%%%%%%%
	%     Cetifications and Awards
	%%%%%%%%%%%%%%%%%%%%%%%%%%%%%%%%%%%%%%


	\section{Awards}\label{sec:certifications-and-awards}
	\runsubsection{ HackHarvard 2023}
	\descript{| Efficiency Boosters Prize}
	% \location{October 2023}
	\begin{tightemize}
		\item Created \href{https://devpost.com/software/snipstudy}{SnipStudy}, a product that helps students extract information from long form video content, such as video lectures, more efficiently.
	\end{tightemize}
	\runsubsection{ HackUMass X}
	\descript{ | Best Use of Twilio}
	% \location{November 2022}
	% \begin{tightemize}
	%     \item Built \href{https://devpost.com/software/who-s-there-wkdheg}{Who's There}, a smart lock that uses a state-of-the-art transformers image captioning model to tell someone what is happening outside of their door
	% \end{tightemize}
	% \location{ July 2020}
	% \begin{tightemize}
	%     \item Given for successful completion of "Deep Learning with Pytorch: Zero to GANs",
	%     a six-week online course offered in collaboration by FreeCodeCamp and JovianAI.
	%     \item Represents about 60 hours of coursework, which required doing weekly assignments, watching
	%     lectures, a course project and a Kaggle Competition
	% \end{tightemize}
	\sectionsep{}
	\runsubsection{\href{https://stamforddatascience.com/hackathon}{Coindesk x TradeBlock Crypto Hackathon}}
	\descript{| 1st Place }
	% \location{ February 2022 }
	% \begin{tightemize}
	%     \item With a 5-person team developed a custom momentum based algorithm that detected rises and falls within
	%     Bitcoin and Ethereum prices with a custom load factor to detect volumes of trades
	%     % \item We faced problems with the data such as invalid/null values and high volatility which needed to be
	%     % accounted for
	% \end{tightemize}
	\sectionsep{}
	%        \runsubsection{Zero To GANs}
	%        \descript{|  Certification, July 2020}
	% \location{ July 2020}
	% \begin{tightemize}
	%     \item Given for successful completion of "Deep Learning with Pytorch: Zero to GANs",
	%     a six-week online course offered in collaboration by FreeCodeCamp and JovianAI.
	%     \item Represents about 60 hours of coursework, which required doing weekly assignments, watching
	%     lectures, a course project and a Kaggle Competition
	% \end{tightemize}
	%        \sectionsep{}

	%        \runsubsection{Zero To Pandas}
	%        \descript{|  Certification, August 2021}
	% \location{August 2021}
	% % \begin{tightemize}
	% %     \item Given for successful completion of "Data Analysis with Python: Zero to Pandas",
	% %     a six-week online course offered in collaboration by FreeCodeCamp and JovianAI.
	% %     \item Represents about 60 hours of coursework, which required doing weekly assignments, watching lectures, and a course project.
	% % \end{tightemize}
	%        \sectionsep{}

	%%%%%%%%%%%%%%%%%%%%%%%%%%%%%%%%%%%%%%
	%     PUBLICATIONS
	%%%%%%%%%%%%%%%%%%%%%%%%%%%%%%%%%%%%%%




\end{minipage}
\end{document}
