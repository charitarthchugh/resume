%%%%%%%%%%%%%%%%%%%%%%%%%%%%%%%%%%%%%%%
% Deedy - One Page Two Column Resume
% LaTeX Template
% Version 1.2 (16/9/2014)
%
% Original author:
% Debarghya Das (http://debarghyadas.com)
%
% Original repository:
% https://github.com/deedydas/Deedy-Resume
%
% IMPORTANT: THIS TEMPLATE NEEDS TO BE COMPILED WITH XeLaTeX
%
% This template uses several fonts not included with Windows/4


% default. If you get compilation errors saying a font is missing, find the line
% on which the font is used and either change it to a font included with your
% operating system or comment the line out to use the default font.
% 
%%%%%%%%%%%%%%%%%%%%%%%%%%%%%%%%%%%%%%
% 
% TODO:
% 1. Integrate biber/bibtex for article citation under publications.
% 2. Figure out a smoother way for the document to flow onto the next page.
% 3. Add styling information for a "Projects/Hacks" section.
% 4. Add location/address information
% 5. Merge OpenFont and MacFonts as a single sty with options.
% 
%%%%%%%%%%%%%%%%%%%%%%%%%%%%%%%%%%%%%%
%
% CHANGELOG:
% v1.1:
% 1. Fixed several compilation bugs with \renewcommand
% 2. Got Open-source fonts (Windows/Linux support)
% 3. Added Last Updated
% 4. Move Title styling into .sty
% 5. Commented .sty file.
%
%%%%%%%%%%%%%%%%%%%%%%%%%%%%%%%%%%%%%%%
%
% Known Issues:
% 1. Overflows onto second page if any column's contents are more than the
% vertical limit
% 2. Hacky space on the first bullet point on the second column.
%
%%%%%%%%%%%%%%%%%%%%%%%%%%%%%%%%%%%%%%


\documentclass[]{resume-template}
\usepackage{fancyhdr}

\pagestyle{fancy}
\fancyhf{}

\begin{document}

%%%%%%%%%%%%%%%%%%%%%%%%%%%%%%%%%%%%%%
%
%     LAST UPDATED DATE
%
%%%%%%%%%%%%%%%%%%%%%%%%%%%%%%%%%%%%%%
\lastupdated{}

%%%%%%%%%%%%%%%%%%%%%%%%%%%%%%%%%%%%%%
%
%     TITLE NAME
%
%%%%%%%%%%%%%%%%%%%%%%%%%%%%%%%%%%%%%%
\namesection{Charitarth Chugh}{}{\href{mailto:contact@charitarth.dev}{contact@charitarth.dev} |
    \urlstyle{same}\href{https://charitarth.dev}{charitarth.dev} | 475.434.6427}

%%%%%%%%%%%%%%%%%%%%%%%%%%%%%%%%%%%%%%
%
%     COLUMN ONE
%
%%%%%%%%%%%%%%%%%%%%%%%%%%%%%%%%%%%%%%

\begin{minipage}[t]{0.33\textwidth}

    %%%%%%%%%%%%%%%%%%%%%%%%%%%%%%%%%%%%%%
    %     EDUCATION
    %%%%%%%%%%%%%%%%%%%%%%%%%%%%%%%%%%%%%%

    \section{Education}\label{sec:education}

    \subsection{UConn\\ Computer Science \&\\
        Mathematics-Statistics}\label{subsec:uconn}
    \location{Expected Graduation: May 2025}
    \vspace{\topsep}
    \subsection{Relevant Coursework}\label{subsec:coursework}
    Data Structures \& Algorithms \textbullet{} Systems Programming \textbullet{} Cybersecurity \textbullet{}\\ Data Manipulation\\  
    \location{Upcoming (Spring 2023):}
    Computer Architecture \textbullet{} Algorithms \textbullet{} Analysis of Experiments
    \sectionsep{}


    %%%%%%%%%%%%%%%%%%%%%%%%%%%%%%%%%%%%%%
    %     Volunteering and Activities
    %%%%%%%%%%%%%%%%%%%%%%%%%%%%%%%%%%%%%%


    \section{Activities}\label{sec:activities}

    \subsection{UConn AI Club}\label{subsec:uconn-ai-club}
    \descript{Secretary: 2021 - 2023}
    - AI Club does workshops,\\showcases, and projects \\around deep learning.\\
    - Responsible for planning and\\ leading weekly meetings\\
    \vspace{\topsep}


    % \subsection{Hack Club}\label{subsec:hack-club}
    % \descript{President: 2020 - 2021}
    % \descript{Vice President: 2019 - 2020}
    % \descript{Board: 2018 - 2019}
    % Explore, share, and learn\\ about
    % new technologies, with \\
    % an emphasis on programming\\
    % and computer hardware.\\ Led and organized weekly meetings.
    % \vspace{\topsep}

    \subsection{Volunteering}\label{subsec:volunteering}
    \descript{Time: 220+ hours}
    Volunteered at Trumbull \\
    Public Library for their\\
    Summer Reading Program
    \sectionsep{}

    %%%%%%%%%%%%%%%%%%%%%%%%%%%%%%%%%%%%%%
    %     SKILLS
    %%%%%%%%%%%%%%%%%%%%%%%%%%%%%%%%%%%%%%


    \section{Skills}\label{sec:skills}
    \subsection{Programming}\label{subsec:programming}
    \location{Python:}
    PyTorch \textbullet{} Transformers \textbullet{} \\
    Plotly \textbullet{} Matplotlib \textbullet{} Pandas \textbullet{}\\
    NumPy \textbullet{} FastAPI \textbullet SQLAlchemy \\
    \location{Other:}
    Flutter \textbullet{} Git \textbullet{} GitHub \textbullet{} SQLite \textbullet{}\\
    Linux \textbullet Docker \textbullet Podman \textbullet {} CI/CD \\
    JavaScript \textbullet{} NodeJS \textbullet{} React \\
    \location{Familiar:}
    Java \textbullet{} Kotlin \textbullet{} Bash \textbullet{} Fish \textbullet{} HTML \textbullet{}\\
    CSS \textbullet{} \LaTeX \textbullet{} OpenAPI
    \vspace{\topsep}
    \subsection{Languages}
    English \textbullet{} Hindi (Speaker)\\
    Spanish (Basic)
    %\location{}
    %%%%%%%%%%%%%%%%%%%%%%%%%%%%%%%%%%%%%%
    %     LINKS
    %%%%%%%%%%%%%%%%%%%%%%%%%%%%%%%%%%%%%%


    \section{Links}\label{sec:links}
    GitHub:// \href{https://github.com/charitarthchugh}{\textbf {charitarthchugh}} \\
    LinkedIn:// \href{https:///www.linkedin.com/in/charitarth}{\textbf {charitarth}} \\
    Twitter:// \href{https://twitter.com/charitarthchugh}{\textbf{@charitarthchugh}}\\
    Kaggle:// \href{https://kaggle.com/charitarth}{\textbf{charitarth}}\\
    Medium:// \href{https://medium.com/@charitarth.chugh}{\textbf{@charitarth.chugh}}\\
    %%%%%%%%%%%%%%%%%%%%%%%%%%%%%%%%%%%%%%
    %
    %     COLUMN TWO
    %
    %%%%%%%%%%%%%%%%%%%%%%%%%%%%%%%%%%%%%%

\end{minipage}
\hfill
\begin{minipage}[t]{0.66\textwidth}

    %%%%%%%%%%%%%%%%%%%%%%%%%%%%%%%%%%%%%%
    %     Projects
    %%%%%%%%%%%%%%%%%%%%%%%%%%%%%%%%%%%%%%

    \section{Projects}\label{sec:projects}

    \runsubsection{Bookie}\label{subsec: Bookie}
    \descript{| Full Stack}
    \location{May 2022-Current}
    \vspace{\topsep}
    \begin{tightemize}
        \item Creating a cross-platform bookmark manager using Fast API, SQLite \& Flutter.
        \item Served as the Lead Developer and Project Manager in a small team
        \item Created CLI interface, API, daemon and was responsible for Python packaging.
        % while also contributing to the creation and design of the Flutter application. In my role as the project manager, I supervised the creation of new features and encouraged standards that promote future
        % maintainability
        \item Supervised the creation of new features and encouraged standards that promote future maintainability
    \end{tightemize}
    \vspace{\topsep}

    \runsubsection{OpinionMining}\label{subsec:opinionmining}
    \descript{| Natural Language Processing}
    \location{October 2021 - Current}
    %\vspace{\topsep}
    \begin{tightemize}
        \item Opinion Mining, also known as Aspect-based Sentiment Analysis (ABSA) is a subfield of sentiment analysis
        where a model detects one or more entities, aspects and opinions within a textual input.
        \item Created a BERT model with a custom head that better detects implicit opinion within a given input
        \item Working to integrate model with a rule based sentiment analysis algorithm
    \end{tightemize}
    \vspace{\topsep}

    \runsubsection{Spam Classification}\label{subsec:pulsar-identification}
    \descript{| Machine Learning}
    \location{June 2020}
    %\vspace{\topsep} % Hacky fix for awkward extra vertical space
    \begin{tightemize}
        \item Created a 97\% accurate classifier using a custom Logistic Regression model made with Numpy, Pandas, and PyTorch for the classification of pulsars in the HTRU1 dataset
        %\item Used Numpy, Pandas, and PyTorch to create a custom Logistic Regression model.
    \end{tightemize}

    \runsubsection{Exo-EDA}\label{subsec:exo-eda}
    \descript{| Data Analysis}
    \location{July - August 2021}
    %\vspace{\topsep} % Hacky fix for awkward extra vertical space
    \begin{tightemize}
        \item In-depth analysis of exoplanet data from the NASA Exoplanet Archive, using Pandas, NumPy,
        Seaborn, and Matplotlib.
        % \begin{itemize}
        %     % \item Found exoplanets that orbit multiple stars
        %     % \item 
        % \end{itemize} 
        \item Retrieved data using a domain-specific API
        \item Cleaned the data and identified potential planets that reside in the habitable zone of their host star
        \item Looked for relative anomalies in the data, such as planets orbiting multiple stars
        \item Found planets with a chance of habitability by

    \end{tightemize}
    \vspace{\topsep}

    % \runsubsection{Pulsar Identification}\label{subsec:pulsar-identification}
    % \descript{| Machine Learning}
    % \location{June 2020}
    % %\vspace{\topsep} % Hacky fix for awkward extra vertical space
    % \begin{tightemize}
    %     \item Created a 97\% accurate classifier using a custom Logistic Regression model made with Numpy, Pandas, and PyTorch for the classification of pulsars in the HTRU1 dataset
    %     %\item Used Numpy, Pandas, and PyTorch to create a custom Logistic Regression model.
    % \end{tightemize}
    % \sectionsep
    \runsubsection{NeatBot}\label{subsec:neatbot}
    \descript{| MLOps}
    \location{June 2022}
    \begin{tightemize}
        \item Created a Discord Bot that detects code languages being used in a code block and replies with the correct syntax highlighting
        \item Deployed to Google Cloud Platform using Docker
    \end{tightemize}


    %\sectionsep


    %\sectionsep
    \sectionsep{}

    %%%%%%%%%%%%%%%%%%%%%%%%%%%%%%%%%%%%%%
    %     RESEARCH
    %%%%%%%%%%%%%%%%%%%%%%%%%%%%%%%%%%%%%%

    %\section{Research}\label{sec:research}
    %\runsubsection{Cornell Robot Learning Lab}
    %\descript{| Researcher}
    %\location{Jan 2014 – Jan 2015 | Ithaca, NY}
    %Worked with \textbf{\href{http://www.cs.cornell.edu/~ashesh/}{Ashesh Jain}} and \textbf{\href{http://www.cs.cornell.edu/~asaxena/}{Prof Ashutosh Saxena}} to create \textbf{PlanIt}, a tool which  learns from large scale user preference feedback to plan robot trajectories in human environments.  
    %\sectionsep

    %\runsubsection{Cornell Phonetics Lab}
    %\descript{| Head Undergraduate Researcher}
    %\location{Mar 2012 – May 2013 | Ithaca, NY}
    %Led the development of \textbf{QuickTongue}, the first ever breakthrough tongue-controlled game with \textbf{\href{http://conf.ling.cornell.edu/~tilsen/}{Prof Sam Tilsen}} to aid in Linguistics research. 
    %\sectionsep

    %%%%%%%%%%%%%%%%%%%%%%%%%%%%%%%%%%%%%%
    %     Cetifications and Awards
    %%%%%%%%%%%%%%%%%%%%%%%%%%%%%%%%%%%%%%


    \section{Certifications and Awards}\label{sec:certifications-and-awards}
    \runsubsection{\href{https://stamforddatascience.com/hackathon}{Coindesk x TradeBlock Crypto Hackathon}}
    \descript{| 1st Place }
    \location{ February 2022 }
    \begin{tightemize}
        \item With a  5-person team developed a custom momentum based algorithm that detected rises and falls within
        Bitcoin and Ethereum prices with a custom load factor to detect volumes of trades
        \item We faced problems with the data such as invalid/null values and high volatility which needed to be
        accounted for
    \end{tightemize}
    \sectionsep{}
    \runsubsection{Zero To GANs}
    \descript{|  Certification, July 2020}
    % \location{ July 2020}
    % \begin{tightemize}
    %     \item Given for successful completion of "Deep Learning with Pytorch: Zero to GANs",
    %     a six-week online course offered in collaboration by FreeCodeCamp and JovianAI.
    %     \item Represents about 60 hours of coursework, which required doing weekly assignments, watching
    %     lectures, a course project and a Kaggle Competition
    % \end{tightemize}
    \sectionsep{}

    \runsubsection{Zero To Pandas}
    \descript{|  Certification, August 2021}
    % \location{August 2021}
    % \begin{tightemize}
    %     \item Given for successful completion of "Data Analysis with Python: Zero to Pandas",
    %     a six-week online course offered in collaboration by FreeCodeCamp and JovianAI.
    %     \item Represents about 60 hours of coursework, which required doing weekly assignments, watching lectures, and a course project.
    % \end{tightemize}
    \sectionsep{}

    %%%%%%%%%%%%%%%%%%%%%%%%%%%%%%%%%%%%%%
    %     PUBLICATIONS
    %%%%%%%%%%%%%%%%%%%%%%%%%%%%%%%%%%%%%%




\end{minipage}
\end{document}
